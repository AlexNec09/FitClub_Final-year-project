\chapter{Tehnologii utilizate}
\section{HTML}

HTML\footnote{Hyper Text Mark-up Language} reprezintă un limbaj de marcare utilizat în componența paginilor web afișate prin intermediul unui browser. Fișierele de tip HTML sunt alcătuite din elemente pereche, unul marcând deschiderea iar perechea sa, închiderea.
Navigatorul folosit de către utilizatori este capabil, astfel, să interpreteze acest limbaj, fară să țină cont de majuscule sau minuscule, afișând rezultatul prin intermediul unei ferestre web, generată automat de browser la accesarea unui domeniu.
\begin{center}
\includegraphics[width=7cm]{./images/html-logo.png}\newline\newline\newline
\end{center}
\textbf{ Componența unui document HTML este:}\newline

\paragraph{• 1} versiunea de HTML folosita
\paragraph{• 2} etichete specifice zonei de head
\begin{lstlisting}[language=html]
	<head> </head>
\end{lstlisting}
\paragraph{• 3} etichete specifice zonei de body
\begin{lstlisting} [language=html]
	<body> </body>
	     sau 
    <frameset> </frameset>
\end{lstlisting}
\section{HTML5}
A fost lansat spre uz public la data de 22 ianuarie 2008, având următoarea actualizare majoră în octombrie 2014. Acest update urmărea îmbunătățirea limbajuui deja existent în raport cu ultimele technologii multimedia apărute. \newline
A fost dezvoltat cu discernământul de a-l putea înțelege cu ușurință atât oamenii cât și device-urile precum browser-ele web, parser-ele etc. fără rigiditatea XHTML-ului. \newline

\textbf{ HTML4 vs HTML5 :}\newline

În comparație cu ediția anterioară de HTML, respectiv HTML4, au fost modificate în mod direct structura, echivalentă marcării, caracteristicile care permit redarea (culori, font etc.), echivalente cu directivele destinate stilului paginilor web și partea de conținut. De asemenea, versiunea nouă  oferă suport în ceea ce privește stocarea de cantități consistente, descărcate de pe web, pe spațiul local de memorie, permițând astfel utilizarea unora dintre
aplicațiile care au web-ul la baza și fară accesul la internet.\newline\newline\newline\newline\newline

\section{CSS}
\label{chap:02}