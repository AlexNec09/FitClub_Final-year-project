\chapter{Introducere}
\section{Descrierea succintă a temei}

Această lucrare constă în dezvoltarea unei platforme web de social networking. Ideea acestei aplicații este de a împărtăși și de a găsi cele mai bune fotografii și videoclipuri. Fiecare profil de utilizator are un număr de urmăritori și urmăritori, reprezentând câte persoane urmăresc și câți alți utilizatori îi urmăresc.\newline

Utilitatea aplicației se rezuma la posibilitatea de a partaja, în timp real, postări ce conțin text și chiar și imagini, însă beneficiază și de un sistem de reacții din partea următorilor, așadar un utilizator care urmărește un alt utilizator poate reacționa prin apreciere sau depreciere la postările acestuia.\newline

Dacă un utilizator dorește să urmărească un alt utilizator, acest lucru este posibil din pagina de profil a respectivului utilizator. Ca și precondiții, este necesară autentificarea cu un cont de utilizator, prin care oricine poate găsi și vizualiza profilul, împreună cu postările recente de pe site. În caz contrar, vizualizarea profilului unui utilizator este în continuare posibila, însă accesul la resursele disponibile va fi limitat, precum faptul că postările respectivului utilizator nu vor putea fi vizibile.\newline

Din punct de vedere al implementării pe partea de server, am ales framework-ul Spring\cite{.springdoc}, deoarece oferă un model cuprinzător de programare și configurare pentru aplicațiile moderne bazate pe Java, fiind, de asemenea, bine documentat, iar dezvoltarea unei aplicații nu necesita cumpărarea unei licențe.\newline

Din punct de vedere al implementării pe partea de client, am ales să folosesc React\cite{.reactdoc}, deoarece dispune de diverse instrumente preinstalate care pot fi utilizate direct pentru dezvoltare. De asemenea, React beneficiază și de o comunitate numeroasa, iar acest lucru încurajează rezolvarea eventualelor blocaje tehnice în timpul dezvoltării aplicației.

\section{Motivarea alegerii temei}

Transformarea digitală are implicații profunde pentru companii, și pentru societate în ansamblu. Această revoluție digitală este atât de importantă, încât unii experți o compară cu nașterea tiparului în urmă cu mai bine de cinci secole. În această nouă eră, canalele digitale se înmulțesc, iar utilizările lor cresc.\newline
 
Rețelele sociale au devenit instrumente de comunicare esențiale. În prezent, companiile trebuie să profite de oportunitățile digitale pentru a-și dezvolta reputația, pentru a-și adapta cultura și pentru a-și fideliza clienții. Creșterea vizibilității mărcii sau împărtășirea noutăților sunt câteva dintre numeroasele avantaje pe care le poate oferi o platformă de social media unei companii.\newline
 
Dornici de web, de noi forme de consum, datorită internetului, clienții compară, împărtășesc, recomandă un produs/serviciu, iar acest lucru modifică strategiile de marketing, comunicare și vânzare.\newline

Pentru mine, aceste fapte reprezintă un interes deosebit, și totodată, acestea au stat la baza alegerii temei, precum și dorință de a realiza o lucrare cât mai practică, și de actualitate.\newline

Platforma se intitulează FitClub și se adresează clienților unei săli de fitness, oferindu-le posibilitatea de a comunica rapid și eficient.


\section{Structura lucrării}

\paragraph{•  Capitolul 1 -}Capitolul curent, cu rol introductiv în ceea ce privește tema aleasă, motivarea alegerii temei, dar și structura lucrării.

\paragraph{•  Capitolul 2 -}Acest capitol prezintă aspecte teoretice ale lucrării precum și o prezentare detaliată a tehnologiilor folosite la implementarea proiectului folosind  atât exemple cât și explicații.

\paragraph{•  Capitolul 3 -}În acest capitol vor fi prezentate detalii cu privire la implementarea propriu-zisă a aplicației, prezentarea modulelor acesteia, cu explicații sugestive ale codului folosit.

\paragraph{•  Capitolul 4 -}Acest capitol va cuprinde prezentarea aplicației în sine, cu accent pe interfața de utilizator. Capitolul va cuprinde, de asemenea, și un ghid de utilizare al aplicației, cu diverse exemple și instrucțiuni de interacțiune.

\paragraph{•  Capitolul 5 -}Concluziile și funcționalități adiționale pentru a spori complexitatea aplicației, în viitor, vor fi cuprinse în cadrul acestui capitol.

\label{chap:01}


