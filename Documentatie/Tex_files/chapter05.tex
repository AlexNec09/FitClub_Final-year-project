\chapter{Concluzii}
Această lucrare a avut ca scop prezentarea unui prototip de platformă online capabil să conecteze rapid membrii unei săli de fitness în vederea colaborării acestora și pentru a-și împărtăși diverse informații, filmulețe și imagini.
Am propus o arhitectură viabilă care să poată gestiona conexiunile utilizatorilor și să trateze corespunzător diverse scenarii care să faciliteze comunicarea în timp real a acestora.\newline
De asemenea, am abordat și aspectele de securitate pentru a proteja utilizatorii și conținutul pe care îl împărtășesc.\newline

Aplicația realizată poate fi îmbunătățită în viitor, atât pentru a-i spori securitatea, cât și pentru a-i simplifica utilizarea. De asemenea, pot fi implementate multe caracteristici noi:
\begin{itemize}
	\item{Înregistrarea să se poată efectua utilizând și alte platforme precum Facebook, Google, etc. (OAuth)}
	\item{Utilizatorii să poată adăuga comentarii în cadrul unei postări, acestea putând fi considerate tot postări, dar care dețin o "postare-părinte".}
	\item{Posibilitatea de a cita o postare în cadrul unei noi postări.}
	\item{Adăugarea unui tab pe pagina de utilizator în cadrul căruia se vor afișa postările preferate ale utilizatorului respectiv.}
	\newline
\end{itemize}

De asemenea, singura comunicare efectivă între utilizatori este cea reprezentată prin pagina principală de postări. În acest context, se poate adăuga și posibilitatea transmiterii mesajelor private între doi utilizatori pentru a oferi mai multă complexitate și a face comunicarea dintre utilizatori mai ușoară și rapidă.\newline

În urma dezvoltării acestei aplicații, am reușit să dobândesc noi informații și cunoștințe, dar și să aprofundez cunoștințele pe care le aveam deja. De asemenea, din dorința de a crea un sistem cât mai stabil, am construit și menținut un cod eficient, reutilizabil și ușor de înțeles, astfel încât să sporesc productivitatea și să reduc timpul pierdut în cazul depanării cazurilor de eroare neașteptate.
\label{chap:05}